\documentclass{article}

% Language setting
% Replace `english' with e.g. `spanish' to change the document language
\usepackage[english]{babel}

% Set page size and margins
% Replace `letterpaper' with `a4paper' for UK/EU standard size
\usepackage[letterpaper,top=2cm,bottom=2cm,left=3cm,right=3cm,marginparwidth=1.75cm]{geometry}

% Useful packages
\usepackage{amsmath}
\usepackage{graphicx}
\usepackage[colorlinks=true, allcolors=blue]{hyperref}


\title{Investigating the effect of child-related transfers in a household bargaining model under limited commitment.}
\author{Joshua Clyne, Joshua Uhlig and Julius}

\begin{document}
\maketitle
\section{Research question}


The scope and composition of child-related transfers is an important policy question across many economies. This project aims to develop a dynamic structural model of household behaviour, and investigates the introduction of various types of child-related transfers on household labour supply, consumption, and welfare (we can hope for but not promise all of these!). The second-earner in a marriage, who is typically a women, faces a trade-off between allocating time to childcare-related tasks, and working, where she can accumulate human capital and effectively self-insure in the case of divorce. Our focus lies in the effect of child-related transfers on the bargaining power of women within households, and how these transfers may impact the value of the outside option in partnerships (i.e. making the prospect of divorce more attractive).
\section{Background literature}
The analysis plans to build upon the limited commitment framework to model household decisions, where bargaining occurs between partners in a marriage. This framework has been increasingly popular to model the effects of policy reforms such as the move from joint to individual taxation, divorce law, and the effect of laws related to alimony and child support (cite). To the best of our knowledge, our project will be the first of its kind to apply the household bargaining framework to investigate child-related transfers. To do this we extend the model proposed by Hallengreen et al. (forthcoming) to include endogenous labour supply, including human capital formation. We then introduce the random arrival of children and child-related transfers, which is effectively a stylised application of the model of Guner et. al (2020), who study the effects of US childcare transfers using a unitary household model. 
\section{Methods}
Our analysis extends the framework laid out in the guide of Hallengreen et al. (forthcoming), by adding endogeous labour supply choices, human capital formation, and the random arrival of children. This adds one discrete and two continuous state variables to their baseline model, and changes the household budget constraint to include various types of child-related transfers, among other changes. We plan to use the iEGM algorithm (Hallengreen et al., 2024) to solve and simulate our model. We have access to baseline code in c++ which we hope to translate into python code, and modify to solve and simulate our model. We do not plan to estimate our model on any dataset, we think the plan we have set out is already highly ambitious...

\section{Model}

In this simple example, singlehood is absorbing and couples choose
individual consumption, $c_{j,t}$ for $j\in\{w,m\}$, public consumption,
$c_{t}$ and individual working hours, $h_{j,t}$..  Individual preferences are of the CES type,
\begin{align*}
U_{j}(c_{j,t},c_{t}, h_{j,t}, h_{t}, n_t) & =\frac{1}{1-\rho_{j}}{\left(\alpha_{1,j}c_{j,t}^{\phi_{j}}+\alpha_{2,j}c_{t}^{\phi_{j}}\right)^{1-\rho_{j}} - \theta_1(n_t)\frac{h_j^{1-\gamma}}{1+\gamma}}
\end{align*}
with the disutility of working depending on the presence of a child $n_t$,
$$
\theta_j = \theta_{0, j} + \theta_{1, j} n_t
$$
and the household budget for a couple is
$$
A_{t}+c_{t}+c_{w,t}+c_{m,t}=RA_{t-1}+Y_{w,t}+Y_{m,t} + T(Y_t + C) + C 
$$
where $A_{t-1}$ is beginning of period $t$ wealth, where $Y_{j,t}$ is labour income of member $j$, which is based on exogenous wages and their endogenous labour supply choice, 
$$
Y_{j,t} = w_{j, t} h{j, t}.
$$

Human capital of each member, $K_{j,t}$ is accumulated from working, which also enters into the wage function,
\begin{align}
log w_{j,t} = \alpha_{j,0} + \alpha_{j,1}K_{j,t}
\end{align}
\begin{align}
K_{j,t+1} = (1-\delta)K_{j,t} + h_{j,t}
\end{align}

If a child is present, the household may receive child-related transfers, 

\begin{align}
C(n_t, h_{1,t}, h_{2,t}, w_{1,t}, w_{2,t}) = C_1(n_t) + C_2(n_t, Y_t) + [C_3(n_t)+C_4(n_t, Y_t)] * 1(h_{1,t} * h_{2,t}>0)
\end{align}

In each period, a couple faces the random arrival of a child. The child arrival is conditional on them being in a couple (show this in math with some indicator function??)
\begin{align}
p(n_t)=&\begin{cases}
\begin{array}{ll}
p_n & \text{if } n_{t}=0\\
0 &  \text{if } n_{t}=1.
\end{array}\end{cases} \\ 
n_{t+1}= &\begin{cases}
\begin{array}{ll}
n_{t}+1 & \text{with probability } p(n_{t})\\
n_{t} & \text{with probability } 1-p(n_{t})
\end{array}\end{cases} \\
\end{align}
In each period, the couple receives
a random value of remaining as a couple, $\psi_{t}$, which follows
a unit-root process,
$$
\psi_{t+1}=\psi_{t}+\varepsilon_{t+1}
$$
where $\varepsilon_{t}\sim iid\mathcal{N}(0,\sigma_{\psi}^{2})$.
 The
state variables for a couple is then $\mathcal{S}_{t}=(\psi_{t},A_{t-1}, n_t)$
besides the bargaining power coming into the period, $\mu_{t-1}$.

\textbf{The value of entering a period as a couple} is then
$$
V_{j,t}^{m}(\psi_{t},A_{t-1},\mu_{t-1})=D_{t}^{\star}V_{j,t}^{m\rightarrow s}(\kappa_{j}A_{t-1})+(1-D_{t}^{\star})V_{j,t}^{m\rightarrow m}(\psi_{t},A_{t-1},\mu_{t-1})
$$
where $\kappa_{j}$ is the share of household wealth member $j$ gets
in case of divorce ($\kappa_{w}+\kappa_{m}=1$). The choice to divorce,
$D_{t}^{\star}$, is discussed below.

\textbf{The value of transitioning into singlehood} is
\begin{align*}
V_{j,t}^{m\rightarrow s}(A_{t-1}) & =\max_{c_{j,t},c_{t}}U_{j}(c_{j,t},c_{t})+\beta V_{j,t+1}^{s}(A_{t})\\
 & \text{s.t.}\\
A_{t} & =RA_{t-1}+Y_{j,t}-c_{t}-c_{j,t} + T(Y_t + C) + C
\end{align*}
where $V_{j,t+1}^{s}(A_{t})=V_{j,t+1}^{m\rightarrow s}(A_{t})$ since
singlehood is absorbing.

\textbf{The value of remaining married} is
\begin{align*}
V_{j,t}^{m\rightarrow m}(\psi_{t},A_{t-1},\mu_{t-1}) & =U_{j}(c_{j,t}^{\star},c_{t}^{\star})+\psi_{t}+\beta\mathbb{E}_{t}[V_{j,t+1}^{m}(\psi_{t+1},A_{t},\mu_{t})]\\
 & \text{s.t.}\\
A_{t} & =RA_{t-1}+Y_{w,t}+Y_{m,t}-(c_{t}^{\star}+c_{w,t}^{\star}+c_{m,t}^{\star}) + T(Y_t + C) + C\\
\psi_{t+1} & =\psi_{t}+\varepsilon_{t+1}
\end{align*}
where $(c_{w,t}^{\star},c_{m,t}^{\star},c_{t}^{\star})$ and $\mu_{t}$
are found along with $D_{t}^{\star}$ in the following way.

Let the solution to a problem of couples, under the condition that
they remain together taking the bargaining power, $\mu$, be
\begin{align*}
\tilde{c}_{w,t}(\mu),\tilde{c}_{m,t}(\mu),\tilde{c}_{t}(\mu) & =\arg\max_{c_{w,t},c_{m,t},c_{t}}\mu v_{w,t}(\psi_{t},A_{t-1},c_{w,t},c_{m,t},c_{t},\mu)\\
 & +(1-\mu)v_{m,t}(\psi_{t},A_{t-1},c_{w,t},c_{m,t},c_{t},\mu)\\
 & \text{s.t.}\\
A_{t} & =RA_{t-1}+Y_{w,t}+Y_{m,t}-(c_{t}+c_{w,t}+c_{m,t})\\
\psi_{t+1} & =\psi_{t}+\varepsilon_{t+1},\:\varepsilon_{t}\sim iid\mathcal{N}(0,\sigma_{\psi}^{2})
\end{align*}
where the value-of-choice given some $\mu$ is
\begin{align}
v_{j,t}(\psi_{t},A_{t-1},\mu,c_{w,t},c_{m,t},c_{t}) & =U_{j}(c_{j,t},c_{t})+\psi_{t}+\beta\mathbb{E}_{t}[V_{j,t+1}^{m}(\psi_{t+1},A_{t},\mu)].
\end{align}

First, solve the unconstrained problem under the assumption that none
of the participation constraints are violated, such that $\mu=\mu_{t}=\mu_{t-1}$.
This gives $\tilde{c}_{w,t}(\mu_{t-1}),\tilde{c}_{m,t}(\mu_{t-1}),\tilde{c}_{t}(\mu_{t-1})$
and individual values of marriage as $v_{j,t}(\psi_{t},A_{t-1},\mu_{t-1},\tilde{c}_{w,t}(\mu_{t-1}),\tilde{c}_{m,t}(\mu_{t-1}),\tilde{c}_{t}(\mu_{t-1}))$. 
\\
\textbf{Checking the Participation Constraint}
Secondly, check the participation constraints for three cases. For
this purpose let 
$$
S_{j,t}(\mu)=S_{j,t}(\psi_{t},A_{t-1},\mu,c_{w,t},c_{m,t},c_{t})=v_{j,t}(\psi_{t},A_{t-1},\mu,c_{w,t},c_{m,t},c_{t})-V_{j,t}^{m\rightarrow s}(\kappa_{j}A_{t-1})
$$
denote the marital surplus of household member $j$. The three cases
are

1. If $S_{j,t}(\mu_{t-1})\geq0$ for both $j=w,m$, they remain married
and keep the bargaining power unchanged. In turn, I  have $\mu_{t}=\mu_{t-1}$,
$(c_{w,t}^{\star},c_{m,t}^{\star},c_{t}^{\star})=(\tilde{c}_{w,t}(\mu_{t-1}),\tilde{c}_{m,t}(\mu_{t-1}),\tilde{c}_{t}(\mu_{t-1}))$,
and $D_{t}^{\star}=0$.
2. If $S_{j,t}(\mu_{t-1})<0$ for both $j=w,m$, they divorce. In turn
$D_{t}^{\star}=1$ and only $V_{j,t}^{m\rightarrow s}(A_{t-1})$ matters.
3. If one household member, say the woman, has a negative marital surplus
while the man has a positive marital surplus, they re-negotiate $\mu_{t}$.
They do so by finding the lowest value $\tilde{\mu}$ that solves $$ \tilde{\mu}:S_{w,t}(\tilde{\mu})=0 $$ 
making her just indifferent between remaining married and divorcing.
If the man also has a positive surplus for this value, $S_{m,t}(\tilde{\mu})\geq0$,
they remain married and increase the bargaining power of the woman.
In turn, $\mu_{t}=\tilde{\mu}$, $(c_{w,t}^{\star},c_{m,t}^{\star},c_{t}^{\star})=(\tilde{c}_{w,t}(\tilde{\mu}),\tilde{c}_{m,t}(\tilde{\mu}),\tilde{c}_{t}(\tilde{\mu}))$,
and $D_{t}^{\star}=0$. If, on the other hand, $S_{m,t}(\tilde{\mu})<0$,
there is no value of $\mu$ that can sustain the marriage, and the
couple sets $D_{t}^{\star}=1$.



\end{document}